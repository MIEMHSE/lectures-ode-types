\documentclass[fontsize=10pt,a4paper,fleqn]{scrreprt} % Класс печатного документа.


%%% Математика
\usepackage{amsmath,amsfonts,amssymb,amsthm} % AMS
\usepackage{mathtools} % Еще AMS
\usepackage{mathtext} % Русские буквы в фомулах

%%% Номера только там, где нужно
% \usepackage{autonum}

%%% Формула в рамке: \boxed{}

%% Apply labels to enumerate's items
\usepackage{enumitem}

% Treating absolutes in nice way
% USAGE: \abs{smth}
%http://tex.stackexchange.com/questions/43008/absolute-value-symbols
\DeclarePairedDelimiter\abs{\lvert}{\rvert}%
\DeclarePairedDelimiter\norm{\lVert}{\rVert}%

% Swap the definition of \abs* and \norm*, so that \abs
% and \norm resizes the size of the brackets, and the
% starred version does not.
\makeatletter
\let\oldabs\abs
\def\abs{\@ifstar{\oldabs}{\oldabs*}}
%
\let\oldnorm\norm
\def\norm{\@ifstar{\oldnorm}{\oldnorm*}}
\makeatother

% ++ arrow with text:
% \underset{text}{arrow}
% \overset{text}{arrow}


% математические символы русской традиции
\let\oldphi\phi \let\phi\varphi \let\varphi\oldphi
\let\oldPhi\Phi \let\Phi\varPhi \let\varPhi\oldPhi
%\let\oldrho\rho \let\rho\varrho \let\varrho\oldrho
\let\oldepsilon\epsilon \let\epsilon\varepsilon  \let\varepsilon\oldepsilon
\let\oldleq\leq \let\leq\leqslant \let\leqslant \oldleq
\let\oldgeq\geq \let\geq\geqslant \let\geqslant \oldgeq

\usepackage{geometry} % Меняем поля страницы
\geometry{left=2cm}% левое поле
\geometry{right=1.5cm}% правое поле \geometry{top=1cm}% верхнее поле
\geometry{bottom=2cm}% нижне
\newcommand*{\hm}[1]{#1\nobreak\discretionary{}% Перенос знаков в формулах
{\hbox{$\mathsurround=0pt #1$}}{}}


\pretolerance 9000
%%% Шрифты
\usepackage[russian]{babel} % Поддержка русского языка.
\usepackage[X2,T2A]{fontenc} % Кодировки
\usepackage[utf8]{inputenc} % Кодировка TeX-файла
\usepackage{cmap} % Поддержка русского в PDF

%%% Шрифты


\usepackage{euler}
\usepackage{hyperref} %Advanced PDF
%% суммы ломают строчки - да!
\everymath{\displaystyle}

%определения:
\newtheorem*{mydef}{Определение}
%теоремки
\newtheorem*{mytm}{Теорема}

\numberwithin{equation}{section}
\titlehead{Moscow Institute of Electronics and Mathematics
           \\12 Malaya Paveletskaya Street
           \\Moscow
           \\Russian Federation}
\author{Pavel Borisov}
\subject{Типы простейших ОДУ}
\title{Types of differential equations}
\date{2014}
\lowertitleback{Типы дифференциальных уравнений}

\begin{document}
\maketitle
\newpage
\chapter{Уравнения первого порядка}
\section{Уравнения с разделяющимися переменными}
\label{sec:divided}

\subsection{Вид}
\begin{align}
  \label{eq:divided}
  \boxed{M(x)N(y)dx + P(x)Q(y)dy = 0} && \boxed{y' = f(x)g(y)}
\end{align}

\subsection{Пример}

\begin{align*}
  xydx + (x+1) dy = 0
\end{align*}

\subsection{Решение}

\begin{enumerate}
\item приводим к виду $M(x)N(y)dx = P(x)Q(y)dy$
\item в левой части делим на $N(y)$, в левой на $P(x)$, разделяем $x$ и $y$ между разными частями уравнения; учитываем решения, упущенные при делении\\
  Получаем вид $\widetilde{Q}(x)dx = \widetilde{P}(y)dy$
%\item проверяем существование перевернутого уравнения (сформировываем перевернутое уравнение из исходного путем замены всех членов уравнения на обратные им)
\item интегрируем обе части уравнения
\item делаем вид y=f(x), если необходимо (выражение легко упрощается)
\item PROFIT!
\end{enumerate}

\section{Однородные уравнения}
\label{sec:homogeneous}

\subsection{Вид}

\begin{align}
\label{eq:homogeneous}
\boxed{M(x,y)dx + N(x,y)dy = 0} && \boxed{y' =f\left(\frac{y}{x}\right)}
\end{align}

\subsection{Пример}
\begin{align*}
xdy = (x+y)dx
\end{align*}

\subsection{Решение}
\begin{enumerate}
\item Проверка на однородность:
  \begin{equation*}
    \left\{
    \begin{array}{ll}
    M(kx, ky) \equiv k^n M(x,y) & n \in \mathbb{R} \\
    N(kx, ky) \equiv k^n M (x, y) & n \text{ is the same as above}
  \end{array}
  \right.
  \end{equation*}
\item Приводим к виду (\ref{eq:homogeneous})
\item Полагаем
  \begin{align*}
    y = tx && dy = xdt + tdx
  \end{align*}
\item Решаем уравнение формы (\ref{eq:divided})
\item возвращаемся к переменному $y$
\item PROFIT!
\end{enumerate}

\section{Уравнения вида y' = f(ax+by+c)}
\label{sec:axbyc}

\subsection{Вид}
\begin{align}
  y' = f(ax+by+c)
\end{align}

\subsection{Решение}
Делаем замену $z = ax + by + c$, тогда $z' = a + by'$. Получаем уравнение
с разделяющимися переменными вида (\ref{eq:divided}).

\subsection{Пример}
\begin{align*}
  y' &= \frac1{x+2y} \\
  x+2y &= z && ax + by +c = z \\
  1 + 2y' &= z' \\
  \frac{z'-1}2 &= \frac1z && \cdot 2z \\
  z'z - z  &= 2 \\
  z'z &= 2+z\\
  \frac{dz}{dz}  &= \frac{2+z}z\\
  \frac{z}{2+z}dz &= dx
\end{align*}
далее решаем (\ref{eq:divided}) и возвращаемся к $x$ и $y$.

\section{Линейные уравнения первого порядка}
\label{sec:lineardiff}

\subsection{Вид}
\begin{align}
  \label{eq:lineardiff}
  \boxed{y' + a(x) y = b(x)}
\end{align}

\subsection{Алгоритм решения}
\begin{enumerate}
\item Решаем $y' + a(x)y = 0$ (соответствующее линейное однородное уравнение по форме (\ref{sec:divided}) (с разделяющимися переменными. В результате получаем $y = C\cdot z (x)$.
\item Применяем метод вариации постоянной:
  \begin{enumerate}
  \item $C = C(x) \Rightarrow y = C(x)\cdot z(x).$
  \item Выражаем $y'$ через $C'(x)$ и подставляем в исходное уравнение
  \item Записываем ответ в удобном виде с зависимой переменной $y$ (или $x$ в некоторых случаях).
  \end{enumerate}
\end{enumerate}

\subsection{Пример решения}
\begin{align}
  &(xy + e^x)dx - xdy = 0 \nonumber && \text{перепишем в виде (\ref{eq:lineardiff})}\\
  &(xy +e^x) - x\frac{dy}{dx} = 0  && x \equiv 0 \text{ является решением } (dx_o = 0; x_o = 0 = const) \nonumber\\
  & x(y-y') + e^x = 0 \label{eq:linear:start} && b(x) = e^x \\
  & x(y-y') = 0 \nonumber \\
  &xy - xy' = 0 \nonumber && \frac{dy}{dx} = y \\
  &\frac{dy}y = dx \nonumber\\
  &x = \ln|y| + C_0 && C_0 \in \mathbb{R}\\
  &e^x = |y|\cdot C_1 && C_1 = e^{C_0} \nonumber \\
  &e^x = y\cdot C_2 && C_2 = \pm C_1 \nonumber \\
  & y = C_3e^x && C_3 = \frac1{e^C2} \label{eq:answ}
  \end{align}
Применим метод вариации постоянной.
\begin{align*}
  y' = C'_3(x)e^x + C_3(x)e^x && \text{выразили } y' \text{ через } C_3'(x)
\end{align*}
\begin{align}
  x(C_3'(x) + C_3(x) - C_3(x))e^x = e^x && \text{подставили } C_3(x) \text{ и } C_3'(x) \text{ в } (\ref{eq:linear:start}) \\
  xC_3'(x) = 1 &&
  C_3'(x) = \frac1x\nonumber \\
  C_3(x) = \ln(x) + C  \label{eq:varansw}
\end{align}

\textbf{Ответ: } $y = e^x(ln|x| + C); x \equiv 0$ (подставили (\ref{eq:varansw} в \ref{eq:answ}))

\section{Уравнение Бернулли}
\label{sec:bernulli}
\subsection{Вид}
\begin{align}
  \boxed{y' = a(x)y + b(x)y^\alpha; \alpha \neq 1 \wedge \alpha \neq 0}
\end{align}

\subsection{Решение}
\begin{enumerate}
\item разделить на $y^\alpha$
\item проанализировать существование решения $y = 0$
\item сделать замену $\frac1{y^{n-\alpha}} = z \Rightarrow z' = (\alpha - 1)\frac{y'}{y^\alpha}$
\item Решить уравнение вида (\ref{eq:lineardiff})
\item PROFIT!!!
\end{enumerate}

\section{Уравнение в полных дифференциалах}
\label{sec:fulldiff}

\subsection{Вид}
\begin{align}
  \boxed{P(x,y)dx + Q(x,y)dy = 0}
\end{align}

\subsection{Алгоритм решения}
Хотим найти функцию $u$, такую, что в левой части уравнения записан её дифференциал, т.е. уравнение имеет вид:
\begin{align*}
  du(x,y) = 0\\
\end{align*}
Тогда его интеграл записывается так:
\begin{align*}
u(x,y) = C\\
  du(x,y) = \frac{\partial u}{\partial x}dx + \frac{\partial u}{\partial y}dy\\
  \boxed{\frac{\partial P}{\partial y} = \frac{\partial Q}{\partial x}}
\end{align*}
\begin{align}
  \begin{cases}
    \frac{\partial u}{\partial x} &= P(x,y)\\
    \frac{\partial u}{\partial y} &= Q(x,y)\label{eq:uy}\\
  \end{cases}
  u(x,y) &= \int{}P(x,y)dx + C(y)
\end{align}
Получаем $C'(y)$ путём вычисления частной производной полученного выражения: $\frac{\partial u}{\partial y}$ и выражаем через имеющееся выражение $Q(x,y)$ (\ref{eq:uy}).

\subsection{Пример}

\begin{align*}
&  \frac{3x^2+y^2}{y^2}dx - \frac{2x^3+5y}{y^3}dy = 0 \\
  \begin{cases}
    P(x,y) =\frac{3x^2+y^2}{y^2}\\
    Q(x,y) = -\frac{2x^3+5y}{y^3}\\
  \end{cases}\\
&  \frac{\partial P}{\partial y} = \frac{\partial Q}{\partial x} = -\frac{6x^2}{y^3}\\
  \begin{cases}
    \frac{\partial u}{\partial x} = \frac{3x^2}{y^2} + 1\\
    \frac{\partial u}{\partial y} = -\frac{2x^3}{y^3} - \frac5{y^2}\\
  \end{cases}\\
&  u(x,y) = \int\left(\frac{3x^2}{y^2} +1 \right)dx +C(y) =  \frac{x^3}{y^2} + x +C(y)\\
& \frac{\partial u}{\partial y} = -\frac{2x^3}{y^3} + C'(y) = -\frac{2x^3}{y^3} - \frac5{y^2}\\
& C'(y) = -\frac5{y^2}\\
& C(y) = \frac5y + C_1, C_1 \in \mathbb{R}\\
&\frac{x^3}{y^2} + x+ \frac5y + C_1 = C_2\\
\end{align*}
Ответ: $ \frac{x^3}{y^2} + x+ \frac5y = C_3.$

\section{Уравнение Клеро}
\label{sec:clerau}

\subsection{Вид и решение}
\begin{align}
  y = xy' + \phi(y')
\end{align}
Пусть $y_x' = \rho$
\begin{align*}
  y &= x\rho+\phi(\rho)\\
  \rho &= x\rho' + \rho + \phi'(\rho)\rho'\\
\end{align*}
\[
\begin{cases}
  y = x\rho + \phi(\rho)\\
  \left[
    \begin{array}{ll}
      \rho' = 0\\
      x + \phi(\rho) = 0
    \end{array}
    \right.
\end{cases}
\]
\[
\left[
\begin{array}{ll}
  \begin{cases}
    \rho = c = const\\
    y = cx + \phi(c)\\
  \end{cases}\\
  \begin{cases}
    x = -\phi'(\rho)\\
    y = x\rho + \phi(\rho)\\
  \end{cases}\\
\end{array}
\right.
\left[
  \begin{array}{ll}
    y = cx+ \phi(c)\\
    \begin{cases}
      x = \phi'(\rho)\\
      y = -\phi'(\rho)\cdot\rho + \phi(\rho)
    \end{cases}
  \end{array}
\right.
\]

\subsection{Пример}
\begin{align}
  y = xy' - (y')^2 \\
  y' = \rho\nonumber\\
  y =  \rho x - \rho^2\label{eq:cl:finaly}\\
  \rho = x \rho' + \rho - 2\rho\cdot \rho'\nonumber\\
  x\rho' - 2\rho\rho'  = 0\nonumber\\
  \rho'(x-2\rho) = 0\nonumber\\
  \begin{cases}
    \left[
      \begin{array}{ll}
        \rho' = 0 \nonumber\\
        x - 2\rho = 0\nonumber\\
      \end{array}
    \right.\\
      y = \rho{}x - \rho^2 (\ref{eq:cl:finaly}) \nonumber
    \end{cases}
\end{align}
\begin{enumerate}
\item
  \begin{align}
    \rho' = 0 \nonumber\\
    p = C && y = Cx - C^2 \nonumber\\
  \end{align}
\item
  \begin{equation*}
    \begin{cases}
      x =  2\rho\\
      y = \rho{}x - \rho^2 = \rho^2
    \end{cases}\\
    y = \frac{x^2}4
  \end{equation*}
\end{enumerate}
{\bfseries Ответ:}
\begin{align*}
\boxed{y = Cx-C^2}\\
\boxed{y=\frac{x^2}4}
\end{align*}

\section{Уравнение Лагранжа}
\label{sec:lagrange}
\subsection{Вид и решение}
\begin{align}
\label{eq:lagrange}
  y = x \psi(\rho) + \phi(\rho)
  \psi(y') \neq y'
\end{align}
Пусть $y'_x = \rho$:
\begin{align*}
&  y = x\psi(\rho) + \phi(\rho)\\
&  \rho = x\psi'(\rho)\rho' + \psi(\rho) + \phi'(\rho)\rho'\\
&  \rho - \psi(\rho) = \frac{d\rho}{dx}(x\psi'(\rho) + \phi'(\rho)) && \cdot\frac{dx}{d\rho} \\
& \frac{dx}{d\rho}(\rho - \psi(\rho)) - x\psi'(\rho) =  \phi'(\rho) && \text{кроме того, } \rho = const \text{ может являться решением, если } d\rho = 0
\end{align*}
Далее решается линейное однородное уравнение (\ref{sec:lineardiff}) ($x$ - функция, $\rho$ - аргумент).

\subsection{Пример}
\begin{align*}
&  y + xy' = 4\sqrt{y'}\\
& y = -xy' + 4\sqrt{y'}\\
\end{align*}
Пусть $y' = \rho$
\begin{align}
&  y = -x\rho+4\sqrt{\rho}\nonumber\\
& \rho = -x\rho' - \rho +4\cdot\frac12\cdot\frac1{\sqrt{\rho}}\cdot\rho'\nonumber\\
& 2\rho = \rho'(-x + \frac2{\rho})\nonumber\\
& 2\rho\frac{dx}{d\rho} + x = \frac2{\rho}\label{eq:lag:linear}
\end{align}
Решаем линейное однородное уравнение вида (\ref{sec:lineardiff}).
\begin{align*}
  a(\rho) = 2\rho && b(\rho) = \frac2{\rho}
\end{align*}
\begin{align}
&  2\rho{dx}{d\rho} = -x\nonumber\\
&  2\frac{dx}x = -\frac12\frac{d\rho}\rho\nonumber\\
&\ln |x| = ln\frac1{\sqrt{|\rho|}} + \ln C_1, C_1 > 0\nonumber \\
&|x| = \frac{C_1}{\sqrt{|\rho|}}\nonumber\\
&x = \frac{C_2}{\sqrt{|\rho|}}, C_2 = \pm C_1\nonumber & \text{опускаем модульные скобки, т.к. } y' \geq 0\\
&x = \frac{C_2(\rho)}{\sqrt{\rho}}\label{eq:lag:linearresult}
\end{align}
Подставляем (\ref{eq:lag:linearresult}) в (\ref{eq:lag:linear}):
\begin{align}
&  2\rho\left(\frac{C_2'(\rho)}{\sqrt{\rho}} - \frac12 \frac{C_2(\rho)}{\rho^{\frac32}}\right) + \frac{C_2(\rho)}{\sqrt{\rho}} = \frac2{\rho}\nonumber\\
&C_2'(\rho) = \frac1\rho\nonumber\\
&C_2(\rho) = \ln |p| + C_3\nonumber\\
\end{align}
\begin{align*}
  \begin{cases}
  &  x = \frac{\ln|p| + C_3}{\sqrt{\rho}}\\
  &  y = -x\rho + 4\sqrt{\rho}
  \end{cases} &&
  \begin{cases}
  &  x = \frac{\ln|p| + C_3}{\sqrt{\rho}}\\
  &  y = -\frac{\ln|p| + C_3}{\sqrt{\rho}}\rho + 4\sqrt{\rho}\\
  &  y \equiv 0
  \end{cases}
\end{align*}


\section{Уравнения вида  x =f(y')}
\label{sec:derarg}
\subsection{Вид и решение}
\begin{align}
  x = f(y')
\end{align}
Пусть $y' = \rho$
\begin{align*}
x = f(\rho)\\
\frac{dy}{dx} = \rho\\
dy = \rho dx = \rho \cdot f'(\rho)d\rho\\
y = \int \rho f'(\rho)d\rho + C
\end{align*}

\subsection{Пример}
\begin{align*}
&  x = (y')^3 + y'\\
&  y' = \rho\\
&  x = \rho^3 + \rho\\
&  \frac{dy}{dx} = \rho\\
&  dy = \rho dx = \rho d(\rho^3+\rho) = \rho(3\rho^2+1)d\rho\\
&  y = \int(3\rho^3+\rho)d\rho = \frac{3\rho^4}4 + \frac{\rho^2}2 + C
\end{align*}

{\bfseries Ответ:}
\begin{equation*}
\left\{
  \begin{array}{ll}
    x = \rho^3 + \rho\\
    y = \frac{3\rho^4}4 + \frac{\rho^2}2 + C & C \in \mathbb{R}
  \end{array}
\right.
\end{equation*}



\section{Уравнения вида y = f(y')}

\subsection{Вид и решение}
\begin{align*}
  y = f(y')&&
  y' = \rho&&
  y = f(\rho)\\
  \frac{dy}{dx} = \rho\\
  \rho = 0 \Rightarrow y' = 0 \Rightarrow y = f(0) \\
  dx = \frac{dy}\rho = \frac{df(\rho)}\rho = \frac{f'(\rho)d\rho}\rho\\
  \begin{cases}
    x = \int\frac{f'(\rho)}\rho d\rho + C\\
    y = f(\rho)\\
    y = f(0)
  \end{cases}
\end{align*}

\chapter{Понижение порядка уравнений}
Цель понижения порядка уравнений \-- сведение уравнений к ОДУ первого порядка
\label{sec:manyderivatives}
\section{В левую часть уравнения явно не входит y}
\subsection{Вид и решение}
\begin{align}
  F(x, y^{(k)}, y^{(k+1)} = 0
\end{align}
Делаем замену: $y^{(k)} = z \Rightarrow y^{(k+1)} = z'$
\begin{align}
  F(x, y', y'') = 0
\end{align}
Пусть $y' = z, y'' = z'$. Пример:
\begin{align*}
&  x^2y'' = (y')^2  && z = y'\\
& x^2z' = z^2 && z = 0\\
&x^2\frac{dz}{dx} = z^2\\
&\frac{dz}{z^2} = \frac{dx}{x^2}\\
&\frac1z = \frac1x + C_1\\
&z =  \frac1{1/x + C_1}\\
\end{align*}
\begin{equation*}
  \left[
    \begin{array}{ll}
z = \frac{x}{C_1 x+1}\\
z =0 \\
\end{array}
\right. \Leftrightarrow
\left[
\begin{array}{ll}
  y' = \frac{x}{C_1x + 1}\\
  y' = 0
\end{array}
\right. \Leftrightarrow
\left[
  \begin{array}{ll}
    y = \int\left(\frac{x}{C_1x+1}\right)dx\\
    y = C\\
  \end{array}
\right.
\end{equation*}
\begin{equation*}
  \left[
  \begin{array}{ll}
    y = \int \frac{x}{C_1x + 1}dx = \int\frac1{C_1}dx - \int \frac{\frac1{C_1}}{C_1x + 1} dx = \frac{x}{C_1} -\ln |C_1x + 1|\\
    C_1 = 0 \Rightarrow z = x \Rightarrow y' = x \Rightarrow y = x^2+C_2
  \end{array}
  \right.
\end{equation*}

\section{В левую часть явно не входит x}

\subsection{Вид и пример}

\begin{align}
  F(y, y', y'') = 0
\end{align}
Пусть $y'_x = \rho(y)$
\begin{align*}
  y''_{xx} = \frac{d}{dx} (y') = \frac{d}{dx}(\rho(y)) = \frac{d}{dy} \rho(y)\cdot\frac{dy}{dx} = \frac{d\rho}{dy}\rho\\
  F(y, \rho, \frac{d\rho}{dy}\rho) = 0\\
  y'' = 2yy'\\
  y'_x = \rho(y)\\
  y''_{xx} = \frac{d\rho}{dy}\rho\\
  \frac{d\rho}{dy}\rho(y) = 2y\rho(y)\\
  \rho = 0 \Rightarrow y = C\\
  d\rho = 2ydy\\
  \rho = y^2 + C_1\\
  \frac{dy}{dx} = y^2 + C_1\\
  \frac{dy}{y^2+C_1} = dx \\
\end{align*}
\begin{equation*}
  x+C_2=\left[
    \begin{array}{ll}
      -\frac1y + C_3 & C_1 = 0\\
      \frac1{\sqrt{C_1}}\arctg\frac{y}{\sqrt{C_1}} & C_1 > 0 \\
      \frac1{2\sqrt{-C_1}} \ln\abs{\frac{y-\sqrt{-C_1}}{y+\sqrt{-C_1}}} & C_1 < 0
    \end{array}
    \right.
\end{equation*}
\textbf{Ответ:}\\
\begin{equation*}
  \left[
  \begin{array}{ll}
    y = C\\
    y = -\frac1{x+C_2} & C_1 = 0\\
    x+C_2 = \frac1{\sqrt{C_1}}\arctg\frac{y}{\sqrt{C_1}} & C_2 > 0\\
    x+C_2 = \frac1{2\sqrt{-C_1}}\ln\abs{\frac{y-\sqrt{C_1}}{y+\sqrt{C_1}}} & C_2 < 0\\
  \end{array}
  \right.
\end{equation*}
\section{Однородные по y и его производным уравнения}

\subsection{Вид}
\begin{align}
  F(x, y, y', y'') = 0
\end{align}
где $F(x, ky, ky', ky'') = k^lF(x,y,y', y'')$\\
Сначала следует рассмотреть существование решений $y = const: dy = 0$.
Пусть $\displaystyle \frac{y'}{y} = u$
\begin{align*}
y' = uy\\
y'' = u'y + uy' = u'y + u^2y = y\\
F(x,y, yu, y(u'+u^2)) = 0\\
y^lF (x,l,u, u'+u^2) = 0\\
\end{align*}

\subsection{Пример}
\begin{align*}
xyy'' - xy'^2 = yy'\\
\end{align*}
Пусть $\frac{y'}{y}=u$:
\begin{align*}
\frac{y'}{y} = u, y \neq 0\\
y' = uy && y'' = (u' + u^2)y'\\
xy^2(u'+u^2) - xy^2u^2 = y^2u\\
x(u'+u^2) - xu^2 = u \\
xu' + xu^2 - xu^2 = u \\
xu' = u\\
u = Cx + C_1\\
\frac{y'}y = Cx && \frac{dy}y = Cxdx\\
\ln|y|= С\frac{x^2}2 + C_1\\
\end{align*}
\begin{equation*}
  \left[
    \begin{array}{ll}
      y = \pm C_2e^{\frac{Cx^2}2} & C_2 = e^{C_1}\\
      y = 0
    \end{array}
    \right.
\end{equation*}
\begin{align*}
\text{\bfseries Ответ:   }  \boxed{y = C_3\cdot e^{\frac{Cx^2}2}, C_3 \in \mathbb{R}, C \in \mathbb{R}}
\end{align*}

\section{Уравнение, левая часть которого представляет собой полный дифференциал}

\subsection{Вид}
\begin{align}
  F(x,y, y', y'') = 0 \\
  \text{ где } F(x, y, y', y'') = \frac{d}{dx} \Phi(x,y,y')\nonumber\\
  \Phi(x,y,y') = C\nonumber
\end{align}

\subsection{Пример}
\begin{align*}
xy'' - y' = x^2yy'\\
\frac{d}{dx} y^2 = 2yy'\\
\left(\frac{y'}x\right)' = \frac12 \left(y^2\right)'\\
\frac{y'}x = \frac12y^2 + C_1\\
\frac{dy}{\frac12y^2+C_1} = xdx && 2\int\frac{2y}{y^2+2C_1} = \int xdx\\
\frac12y^2 + C_1 = 0 \Rightarrow y = \pm\sqrt{-2C_1}
\end{align*}

\begin{equation*}
  \left[
    \begin{array}{ll}
      \frac{x^2}2 + C = \frac2{C_2}\arctg \frac{y}{C_2} & C_1 > 0, C_2 = \sqrt{2C_1}\\
      \frac{x^2}2 + C = \frac1{C_2} \ln\abs{\frac{y-C_2}{y+C_2}} & C_1 < 0, C_2 =\sqrt{-2C_1}\\
    \end{array}
    \right.
\end{equation*}

\chapter{Линейные уравнения с постоянными коэффициентами}

\section{Однородные (с правой частью = 0)}
\subsection{Вид}
\begin{align}
\label{eq:linearhomo}
  a_0y^{(n)} + a_1y^{(n-1)} + \ldots + a_ny = 0
\end{align}
(\ref{eq:linearhomo}) в дальнейшем используется как ссылка на левую часть данного выражения.

\subsection{Решение}
\label{sec:solvelinearhomo}
\begin{enumerate}
\item Составить характеристическое уравнение вида $a_0\lambda^n + a_1\lambda^{n-1} + \ldots + a_{n-1}\lambda + a_{n} = 0$, где $a_0, a_1, \ldots, a_n - const \in \mathbb{R}$
 и найти его корни и кратность каждого корня.
\item Общим решение будет сумма из слагаемых
  \begin{itemize}
  \item $C_je^{\lambda_ix}$ для каждого $j$-го простого вещественного корня уравнения
  \item $(C_{m+1} + C_{m+2}x + C_{m+3}x^2 + \ldots + C_{m+k}x^{k-1})e^{\lambda_ix}$ для каждого $i$-го вещественного корня уравнения с кратностью $k$. \label{nsec}
  \item $C_{m+1}e^{\alpha x}\cos\beta x + C_{m+2}e^{\alpha x}\sin\beta x$ для каждой пары простых комплексных корней уравнения вида $\alpha \pm \beta i$
  \item $P_{k-1}(x)e^{\alpha x}\cos \alpha x + Q_{k-1}(x)e^{\alpha x}\sin \beta x$ для каждой пары комплексных корней кратности $k$ вида $\alpha \pm \beta i$ с многочленами как в (\ref{nsec}).
  \end{itemize}
\end{enumerate}

\section{Неоднородные тип 1}

\subsection{Вид}
\begin{align}
  \label{eq:nonlinearhomo1}
  (\ref{eq:linearhomo}) =  (b_o + b_1x + \ldots + b_mx^m)\cdot e^{\alpha x}
\end{align}

\subsection{Решение}
\label{sec:solvenonhomotype1}

\begin{enumerate}
\item Для левой части уравнения, приравненной к 0, выполнить (\ref{sec:solvelinearhomo}) (найти общее решение) \label{findgeneralsolutionforhomo}
\item Для правой части уравнения найти общий  вид многочлена $y_1 = x^sQ_m(x)e^{\Upsilon x}$ (частное решение неоднородного уравнения), где:
  \begin{itemize}
  \item $\Upsilon$ - коэффициент перед $x$ в степени экспоненты в данной правой части (или $0$, если нет экспоненты).
  \item $s$ - кратность корня $\Upsilon$ характеристического уравнения, полученного на шаге (\ref{findgeneralsolutionforhomo}) (или $0$, если такого корня нет)
  \item $m$ - Степень многочлена $Q$, совпадающая со степенью многочлена перед $e$ в данной правой части
  \item $Q(x)$ - многочлен в общем виде ($a_0 + a_1^x + \ldots$)
  \end{itemize} \label{eq:findpolynonhomo}
\item Подставить многочлен из предыдущего шага в левую часть дифференциального уравнения (\ref{eq:nonlinearhomo1}), предвадительно найдя необходимые производные.
\item Приравнять коэффициенты при степенях многочленов в левой и правой части и найти коэффициенты многочлена частного решения,
\item Сложить общее решение с полученным на шаге \ref{eq:findpolynonhomo} частным решением
\end{enumerate}

\section{Неоднородные тип 2}

\subsection{Вид}
\begin{align}
  (\ref{eq:linearhomo}) = e^{\alpha x}(P(x)\cos \beta x + Q(x)\sin\beta x)
\end{align}

\subsection{Решение}
Аналогично (\ref{sec:solvenonhomotype1}), но многочлен частного решения имеет вид:

\end{document}
